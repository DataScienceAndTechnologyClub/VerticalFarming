\documentclass[a4paper]{article}
\usepackage[utf8]{inputenc}
\usepackage{enumitem}
\usepackage{hyperref}
\usepackage{siunitx}

\hypersetup{
    colorlinks=true,
    linkcolor=blue,
    filecolor=magenta,      
    urlcolor=cyan,
}

\setlist{noitemsep,topsep=0pt}
\setlength{\parindent}{0pt}
\setlength{\parskip}{3pt}

\title{Vertical Farm - Tasks}
\author{Patrick Schaffner\\patrick.schaffner@unisg.ch}
\date{\today}

\begin{document}
\maketitle

\section{Frame}
\subsection{Construction}
\subsection{Electronics Integration}


\section{Automation}

These tasks require programming a microcontroller,
such as the Arduino or LoPy, and wiring up electrical components.
Only rudimentary knowledge of programming and electrical engineering is required
as step-by-step instructions are readily available on the internet.

Some facts about writing code that cannot be stressed enough:
\begin{itemize}
 \item Programming is an iterative task.
  Try to start out as simple as possible, test your code thoroughly to
  identify errors early on, and only then add more functionality (which again should be
  tested immediately). Writing the final code in one go and then trying to figure
  out where the errors are is rarely successful, especially when electrical components are involved.
  If you have troubles finding an error, try simplifying your code until you have a \href{https://en.wikipedia.org/wiki/Minimal_Working_Example}{minimal working example}.
  The most promising strategy for getting help on programming websites is to post a minimal working example.
 \item \href{https://codeahoy.com/2016/04/30/do-experienced-programmers-use-google-frequently/}{Using Google
  is essential to software development.} If you use Google once for every line of code, you're probably
  doing it right. There are too many details in programming to remember and there's no need to reinvent the wheel.
  Every task here has already been done by someone else, and every error you'll encounter has already been
  asked about somewhere. The difficult thing is
  \href{https://knightlab.northwestern.edu/2014/03/13/googling-for-code-solutions-can-be-tricky-heres-how-to-get-started/}{using Google efficiently}.
\end{itemize}


\subsection{Sensors}

Several sensors need to be wired up to a microcontroller for reading the state
of the vertical farm. These tasks require following instructions to build an
electrical circuit (using a breadboard, no soldering) and using the microcontrollers
built-in functions to read voltage signals. Only little computation is necessary.

Additionally usage and necessary maintenance task shall be documented.


\subsubsection{pH-Meter}

Required items and resources:
\begin{itemize}
 \item Arduino or LoPy (advanced)
 \item Analog pH sensor kit for Arduino
 \
 \item Breadboard \& jumper wires
 \item pH calibration solution
 \item \href{https://www.dfrobot.com/wiki/index.php/PH_meter(SKU:_SEN0161)}{pH meter manual}
 \item \textit{Advanced:} External 5V DC power supply.
\end{itemize}

Instead of an Arduino a LoPy microcontroller may be used. This will be more difficult and
requires additional circuit elements (not documented here) as the LoPy runs on 3.3V while
the sensor needs and returns 5V. In that case the external 5V power supply must be used and
a \href{https://www.allaboutcircuits.com/tools/voltage-divider-calculator/}{voltage divider}
consisting of 2 resistors must be used to rescale the sensor output from 5V to 3.3V.

Instructions:
\begin{itemize}
 \item Connect the sensor to the microcontroller's analog input pin.
 \item Read voltage values from the input pin.
 \item Convert voltage to pH value.
 \item Write function reading and returning pH values from the sensor.
 \item Calibrate sensor and add corrections to the code (get \textit{correct} pH values).
 \item \textit{Advanced:} Increase accuracy by powering the sensor with an external power supply instead of the microcontroller.
\end{itemize}

The manual includes all necessary circuit schematics, program code, and instructions to complete this task.
Note that the given program code is more complicated/sophisticated than necessary.
Breaking it down to the important bits would be a plus.

\textit{Advanced:} Insert an external power supply by
\begin{itemize}
 \item connecting the power supply's negative terminal to the microcontroller's ground,
 \item disconnect the sensor's positive terminal from the microcontroller and connect it to the power supply's positive terminal.
\end{itemize}


\subsubsection{Temperature / Humidity Sensor}

Required items and resources:
\begin{itemize}
 \item Arduino or LoPy (advanced)
 \item DHT11 digital temperature \& humidity sensor
 \item 5k\si{\ohm} resistor
 \item Breadboard \& jumper wires
 \item \href{https://learn.adafruit.com/dht}{Adafruit DHT tutorial}
 \item \href{http://www.micro4you.com/files/sensor/DHT11.pdf}{DHT11 manual}
 \item \href{https://github.com/niesteszeck/idDHT11}{idDHT11}
 \item \href{http://playground.arduino.cc/Main/DHTLib}{DHTLib}
 \item \href{https://github.com/johnmcdnz/LoPy-DHT-transmission}{Example code for LoPy}
\end{itemize}

Instead of an arduino a LoPy microcontroller may be used. However, there might not already exist
software libraries and you'll have to implement the highly time-critical protocol for yourself
(see example code).

Instructions:
\begin{itemize}
 \item Connect the sensor to the arduino according to the circuit schematics given in the tutorial/manual.
 \item Include a DHT software library to your arduino project.
 \item Try reading temperature and humidity values through the functions provided by the included library.
 \item Write a function that calls the library and returns temperature/humidity data.
\end{itemize}

The tutorial and manual show how to wire up the sensor.
The manual also outlines the protocol used to communicate with it,
but there already exist readily available arduino software libraries for that.
Check the tutorial to see how to include software libraries to your project.

Two alternative libraries are listed above in case the tutorial doesn't work.
See \href{https://www.arduino.cc/en/Guide/Libraries#toc5}{this} on how to manually include libraries.
Libraries always include example code in the ``examples'' folder.


\subsubsection{Light Sensor}

Required items and resources:
\begin{itemize}
 \item Arduino or LoPy
 \item Photocell (light sensor)
 \item 10k\si{\ohm} resistor (or higher)
 \item Breadboard \& jumper wires
 \item \href{https://learn.adafruit.com/photocells/using-a-photocell}{Adafruit photocell tutorial}
\end{itemize}

Instructions:
\begin{itemize}
 \item Wire up photocell as outlined in the tutorial. Connect it to an analog input pin.
 \item Read analog voltage values from input pin.
 \item Convert it to a luminosity metric (lux if possible, or just ``bright'', ``dim'', and ``dark'').
 \item Write a function that reads values from the sensor and returns the luminosity metric.
\end{itemize}


\subsubsection{Water Level Meter}

Required items and resources:
\begin{itemize}
 \item Arduino or LoPy
 \item Water level sensor
 \item Breadboard \& jumper wires
 \item \href{http://scidle.com/how-to-use-a-water-level-sensor-module-with-arduino/}{Water level sensor tutorial}
\end{itemize}

Instructions:
\begin{itemize}
 \item Wire positive and negative pins to 5V and GND.
 \item Wire the output pin to an analog input pin on the Arduino.
 \item Read analog values from the input pin. Higher values mean more water.
 \item Measure water level. Create a graph with water level on the X axis and input pin voltage on the Y axis.
 \item Write a function with no arguments that reads voltage values from the sensor and returns the water level in cm.
\end{itemize}


\subsubsection{Electrical Conductivity Meter}
\subsection{Pumps}
\subsubsection{Dosage Pumps}
\subsubsection{Main Pump}
\subsection{Lights}
\subsection{LoRa Communication}
\subsection{Control}
\section{Server}
\subsection{LoRa Communication}
\subsection{Frontend}
\subsubsection{Status Report}
\subsubsection{Command Panel}
\subsubsection{Device Management}
\subsection{Alerts}

\end{document}
